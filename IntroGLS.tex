\documentclass[Chap1main.tex]{subfiles}
\begin{document}
\section{Generalized Least Squares}


 generalized least squares (GLS) is a technique for estimating the unknown parameters in a linear regression model. 
 The GLS is applied when the variances of the observations are unequal (heteroscedasticity), or when there is a certain degree of correlation between the observations. 
 In these cases ordinary least squares can be statistically inefficient, or even give misleading inferences.
 
 
 
   \[ Y = X\beta + \varepsilon, \qquad \mathrm{E}[\varepsilon|X]=0,\ \operatorname{Var}[\varepsilon|X]=\Omega.\]
  


\subsection{Introduction to Generalized Least Squares}
 \begin{equation}
 \boldsymbol{y}_i = \boldsymbol{X}_i\boldsymbol{\beta} + \boldsymbol{\epsilon}_i
 \end{equation}

Estimation under this model has been studied extensively in the linear regression model.

%-------------------------------------------------------------------Simplifying GLS by KH -%
\end{document}
