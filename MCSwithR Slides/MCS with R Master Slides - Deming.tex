\documentclass[compress]{beamer}        % [compress] (written before {beamer} <=> navigation bar one line, all subsections in 1 line instead of 2

% Setup appearance:
\usetheme{CambridgeUS}
%	AnnArbor | Antibes | Bergen |
%	Berkeley | Berlin | Boadilla |
%	boxes | CambridgeUS | Copenhagen |
%	Darmstadt | default | Dresden |
%	Frankfurt | Goettingen |Hannover |
%	Ilmenau | JuanLesPins | Luebeck |
%	Madrid | Malmoe | Marburg |
%	Montpellier | PaloAlto | Pittsburgh |
%	Rochester | Singapore | Szeged |
%	Warsaw
%

\useoutertheme[footline=authorinstitute,subsection=false]{miniframes}
\usecolortheme{whale}

%	albatross | beaver | beetle |
%	crane | default | dolphin |
%	dove | fly | lily | orchid |
%	rose |seagull | seahorse |
%	sidebartab | structure |
%	whale | wolverine


\setbeamertemplate{footline}
{
  \hbox{%
  \begin{beamercolorbox}[wd=.25\paperwidth,ht=2.25ex,dp=1ex,center]{title in head/foot}%
    \usebeamerfont{date in head/foot}\insertshortauthor
  \end{beamercolorbox}%
  \begin{beamercolorbox}[wd=.5\paperwidth,ht=2.25ex,dp=1ex,center]{date in head/foot}%
    \usebeamerfont{title in head/foot}\insertshortinstitute
  \end{beamercolorbox}%
  \begin{beamercolorbox}[wd=.25\paperwidth,ht=2.25ex,dp=1ex,center]{title in head/foot}%
    \usebeamerfont{date in head/foot}
    \insertframenumber{} / \inserttotalframenumber
  \end{beamercolorbox}}%
  \vskip0pt%
}

%\setbeamercolor{titlelike}{parent=structure}
%\setbeamercolor{structure}{fg=beamer@blendedblue}
%% \useinnertheme{rounded}
%\setbeamerfont{block title}{size={}}
%\usefonttheme[onlylarge]{structurebold}   % title and words in the table of contents bold
%\setbeamerfont*{frametitle}{size=\normalsize,series=\bfseries}
\setbeamertemplate{navigation symbols}{}
\setbeamercolor{frametitle}{parent=boxes, bg=white}


% Standard packages

\usepackage[english]{babel}
\usepackage[latin1]{inputenc}
\usepackage{times}
\usepackage[T1]{fontenc}
\usepackage{amsbsy}         % for \boldsymbol command (bold in math mode)
\usepackage{amsfonts, amssymb}
\usepackage{epsfig}
\usepackage{color}
\definecolor{camblue}{RGB}{26,26,89}
\definecolor{Rblue}{RGB}{0,255,255}
\definecolor{Rdarkblue}{RGB}{0,0,255}
\definecolor{Rgreen}{RGB}{0,205,0}
\newcommand{\tcb}{\textcolor{beamer@blendedblue}}
\newcommand{\tcbb}{\textcolor{camblue}}
\newcommand{\tcr}{\textcolor{red}}
\newcommand{\tcg}{\textcolor{gray}}
\newcommand{\tcRg}{\textcolor{Rgreen}}
\newcommand{\tcRdb}{\textcolor{Rdarkblue}}
\newcommand{\tcRb}{\textcolor{Rblue}}
\newcommand{\sq}{\begin{eqnarray}}
\newcommand{\fq}{\end{eqnarray}}
\newcommand{\bp}{$\bullet$\:}


%%%%%%%%%%%%%%%%%%%%%%%%%%%%%%%%%%%%%%%%%%%%%%%%%%%%%%%%%%%%%%%%%%%%%%%%%%%%%%%%%%%%%%%%%%%%%
% THIS IS WHERE THE DOCUMENT BEGINS


%\setbeamercovered{transparent}   % overlays with light grey 1st slide
\title
{
{\huge Title\\[0.3cm] }
}
\author[Kevin O'Brien]{{\bf Author}}
\institute[University of Limerick, Maths \& Stats Dept]{}
\date{}

\begin{document}



%----------------------------------------------------------- %
\section{Regression Methods - "mcr" and Deming regression}

\begin{frame}
\frametitle{Regression Techniques for MCS}

\Large
\begin{itemize}
\item Deming ( and Orthonormal ) Regression
\item The \textbf{mcr} package
\end{itemize}
\end{frame}

\begin{frame}
\frametitle{Deming Regression}
\begin{itemize}
\item Conventional regression models are estimated using the ordinary
least squares (OLS) technique, and are referred to as `Model I
regression' in some papers. 
\item A key feature of Model I
models is that the independent variable(X) is assumed to be measured
without error. 
\item As often pointed out in several papers, including Bland and Altman [1], this assumption invalidates simple linear
regression for use in method comparison studies, as both methods
must be assumed to be measured with error.
\end{itemize}

\end{frame}
%%%%%%%%%%%%%%%%%%%%%%%%%%%%%%%%%%%%%%%%%%%%%%%%%%%%%%%%%%%%%%%%%%%%%%%%%%%%%%%%%%%%%%

\begin{frame}
\frametitle{Deming Regression}
\large
\begin{itemize}
\item The use of regression models that assumes the presence of error in
both variables $X$ and $Y$ have been proposed for use instead.

\item These methodologies are collectively
known as `Error in Variables Regression' or `Model II regression'. They differ from OLS regression in the method used to estimate the parameters of the regression.
 \end{itemize}

\end{frame}
%%%%%%%%%%%%%%%%%%%%%%%%%%%%%%%%%%%%%%%%%%%%%%%%%%%%%%%%%%%%%%%%%%%%%%%%%%%%%%%%%%%%%%

\begin{frame}
\frametitle{Deming Regression}
\large
\begin{itemize}

\item Regression estimates depend on formulation of the model. A
formulation with one method considered as the $X$ variable will
yield different estimates for a formulation where it is the $Y$
variable. 

\item With Model I regression,the models fitted in both cases
will entirely different and inconsistent.  However with Model II
regression, they will be consistent and complementary.

\item The most commonly encounted approach is called "Deming Regression".

\end{itemize}

\end{frame}


%%%%%%%%%%%%%%%%%%%%%%%%%%%%%%%%%%%%%%%%%%%%%%%%%%%%%%%%%%%%%%%%%%%%%%%%%%%%%%%%%%%%%%

%---------------------------------------------------------- %
\subsection{Error in Variables Deming Regression}
\begin{frame}


\frametitle{Deming Regression}
\begin{itemize}
\item Both Variables are assumed to have attendant measurement error.
\item The ratio of the variances of measurement errors for the respective methods is known as the \textbf{Variance Ratio}, denoted $\lambda$.
\item Orthonormal Regression is used to describe the case where Variance Ratio is equal to 1. (i.e. variances are assumed to be equal).
\item Deming Regression describes the case where the Variance Ration is specified at an value other than 1.
\item Dunn [9] cautions against using the approach as there is no straightforward method for determining an estimate for $\lambda$.
\end{itemize}

\end{frame}


%%%%%%%%%%%%%%%%%%%%%%%%%%%%%%%%%%%%%%%%%%%%%%%%%%%%%%%%%%%%%%%%%%%%%%%%%%%%%%%%%%%%%%
\begin{frame}
\frametitle{Deming Regression}
\large
\begin{itemize}
\item As with conventional regression methodologies, Deming's regression
calculates an estimate for both the slope and intercept for the
fitted line, and standard errors thereof. 
\item Therefore there is
sufficient information to carry out hypothesis tests on both
estimates, that are informative about presence of \textbf{fixed} and
\textbf{proportional} bias.
\end{itemize}
\end{frame}
%%%%%%%%%%%%%%%%%%%%%%%%%%%%%%%%%%%%%%%%%%%%%%%%%%%%%%%%%%%%%%%%%%%%%%%%%%%%%%%%%%%%%%
\begin{frame}
\frametitle{Deming Regression}
\large
\begin{itemize}
\item A $95\%$ confidence interval for the intercept estimate can be
used to test the intercept, and hence fixed bias, is equal to
zero. 
\item This hypothesis is accepted if the confidence interval for
the estimate contains the value $0$ in its range. Should this be,
it can be concluded that fixed bias is not present. 
\item Conversely, if
the hypothesis is rejected, then it is concluded that the
intercept is non zero, and that fixed bias is present.
\end{itemize}
\end{frame}
%%%%%%%%%%%%%%%%%%%%%%%%%%%%%%%%%%%%%%%%%%%%%%%%%%%%%%%%%%%%%%%%%%%%%%%%%%%%%%%%%%%%%%
\begin{frame}
\frametitle{Deming Regression}
\large
\begin{itemize}
\item Testing for proportional bias is a very similar procedure. The
$95\%$ confidence interval for the slope estimate can be used to
test the hypothesis that the slope is equal to $1$. 
\item This
hypothesis is accepted if the confidence interval for the estimate
contains the value $1$ in its range. 
\item If the hypothesis is
rejected, then it is concluded that the slope is significant
different from $1$ and that a proportional bias exists.
\end{itemize}
\end{frame}
%%%%%%%%%%%%%%%%%%%%%%%%%%%%%%%%%%%%%%%%%%%%%%%%%%%%%%%%%%%%%%%%%%%%%%%%%%%%%%%%%%%%%%
\begin{frame}

\frametitle{Deming Regression}
\large
\begin{itemize}
\item Deming's Regression suffers from some crucial drawback. Firstly it
is computationally complex, and it requires specific software
packages to perform calculations.\item Secondly it is uninformative
about the comparative precision of two methods of measurement.
Most importantly \item Carol andl Rupert [11] states that Deming's
regression is acceptable only when the precision ratio ($\lambda$,
in their paper as $\eta$) is correctly specified ,but in practice
this is often not the case, with the $\lambda$ being
underestimated.
\end{itemize}
\end{frame}
\end{document}

%%%%%%%%%%%%%%%%%%%%%%%%%%%%%%%%%%%%%%%%%%%%%%%%%%%%%%%%%%%%%%%%%%%%%%%%%%%%%%%%%%%%%%
\begin{frame}
For convenience, a new data set shall be introduced to demonstrate
Demings regression. Measurements of transmitral volumetric flow
(MF) by doppler echocardiography, and left ventricular stroke
volume (SV) by cross sectional echocardiography in 21 patients
withour aortic valve disease are tabulated in \alert{zhang}. This
data set features in the discussion of method comparison studies
in \alert[p.398]{AltmanBook} .

\end{frame}
%%%%%%%%%%%%%%%%%%%%%%%%%%%%%%%%%%%%%%%%%%%%%%%%%%%%%%%%%%%%%%%%%%%%%%%%%%%%%%%%%%%%%%
\begin{frame}

% latex table generated in R 2.6.0 by xtable 1.5-5 package
% Tue Sep 01 13:31:17 2009
\begin{table}[h!]
\begin{center}
\begin{tabular}{|c|c|c||c|c|c||c|c|c|}
  \hline
 Patient & MF  & SV  & Patient & MF  & SV  & Patient & MF  & SV \\
 &($cm^{3}$)&  ($cm^{3}$) & &($cm^{3}$)&  ($cm^{3}$) & &($cm^{3}$)&  ($cm^{3}$)
 \\
  \hline
1 & 47 & 43 &  8 & 75 & 72 &  15 & 90 & 82 \\
  2 & 66 & 70 & 9 & 79 & 92 &  16 & 100 & 100 \\
  3 & 68 & 72 & 10 & 81 & 76 & 17 & 104 & 94 \\
  4 & 69 & 81 & 11 & 85 & 85 &  18 & 105 & 98 \\
  5 & 70 & 60 & 12 & 87 & 82 & 19 & 112 & 108 \\
  6 & 70 & 67 & 13 & 87 & 90 & 20 & 120 & 131 \\
  7 & 73 & 72 & 14 & 87 & 96 &  21 & 132 & 131 \\

   \hline
\end{tabular}
\caption{Transmitral volumetric flow(MF) and left ventricular
stroke volume (SV) in 21 patients. (Zhang et al 1986)}
\end{center}
\end{table}
\newpage
%\begin{figure}[h!]
%  % Requires \usepackage{graphicx}
%  \includegraphics[width=130mm]{ZhangDeming.jpeg}
%  \caption{Deming Regression For Zhang's Data}\label{ZhangDeming}
%\end{figure}

\end{frame}
%%%%%%%%%%%%%%%%%%%%%%%%%%%%%%%%%%%%%%%%%%%%%%%%%%%%%%%%%%%%%%%%%%%%%%%%%%%%%%%%%%%%%%

%%%%%%%%%%%%%%%%%%%%%%%%%%%%%%%%%%%%%%%%%%%%%%%%%%%%%%%%%%%%%%%%%%%%%%%%%%%%%%%%%%%%%%
\end{document}



\begin{frame}

\frametitle{Deming's Regression}
\begin{itemize}
% \item The most commonly known Model II methodology is known as Deming's Regression, (also known an Ordinary Least Product regression). Deming regression is recommended by \alert{CornCoch} as the preferred Model II regression for use in method comparison studies. 
\item As previously noted, the Bland Altman Plot is
uninformative about the comparative influence of proportional bias
and fixed bias. Deming's regression provides independent tests for
both types of bias.

\item For a given $\lambda$, \alert{Kummel} derived the following
estimate for the Demimg regression slope parameter. ($\alpha$ is
simply estimated by using the identity
$\bar{Y}-\hat{\beta}\bar{X}$.)
\begin{equation}
\hat{\beta} =\quad \frac{S_{YY} - \lambda S_{XX}+[(S_{YY} -
\lambda S_{XX})^{2}+ 4\lambda S^{2}_{XY}]^{1/2}}{2S_{XY}}
\end{equation}
\end{itemize}
\end{frame}